
\section{Computation}
I refer to \citet{farmer2017} for the discretization of non-linear Markov process. The implementation of nested fixed point algorithm is explained below.

\subsection{Preparation}
\paragraph{Profit function}

I normalize the productivity with the constant $\mu_{0}$ in the
empirical model. From the estimation equation, we can write the profit
as:
\begin{equation}
\hat{r}_{it}\left(\hat{\phi}_{it}\right)=\hat{\mu}_{0}+\hat{\mu}_{t}+\left(1-\hat{\sigma}_{j}\right)\hat{\rho}_{0}+\left(1-\hat{\sigma}_{j}\right)\hat{\beta}_{k}k_{it}+\left(1+\hat{\theta}_{j}\right)\hat{\beta}_{a}a_{it}-\left(1+\hat{\theta}_{j}\right)\hat{\phi}_{it}
\end{equation}
It follows that the profit can be calculated as 
\begin{equation}
\pi_{it}\left(\hat{\phi}_{it}\right)=-\frac{1}{\hat{\theta}_{j}}\exp\left(\hat{r}_{it}\left(\hat{\phi}_{it}\right)\right)
\end{equation}

\paragraph{Choose the grid points}

We perform the computation by industry. Several coefficients are specific
to each industry: $\mu_{0}$, $\rho_{0}$, and $\sigma_{j}$. Note
that we discretize the productivity into 200 grid points, and age
into 4 groups. To implement the estimation, I use the trapezoid method
to discretize the capital space in evenly distributed 100 points.
Therefore, we are encountered with $100\times4\times4=1600$ types
of firms. For each type of firm, we solve the value function for $100\times2=200$
states. In the end, we solve $1600\times200=320000$ value functions.
We compute the value function by industry by industry. 

\subsection{Inner loop: Value function iteration}

Given $\left(\phi_{it},\,k_{it},\,a_{it}\right),$ we use $V_{d}$
to denote $V\left(\phi_{it},d_{it-1}=d;\,k_{it},a_{it}\right)$.
We also define $\gamma_{it}^{d}\equiv\gamma_{it}\left(d_{it-1}=d,k_{it};\,\kappa^{m},\,\kappa^{s}\right)$,
for $d\in\left\{ 0,\,1\right\} $, and $\kappa\equiv\left(\kappa^{s},\,\kappa^{m}\right)$
be the parameter to be estimated. Employing the exponential distribution,
we can express the value function as 
\begin{align*}
V_{d} & =\Pi_{it}\left(\phi_{it},\mathbf{S}_{it}\right)+\int_{0}^{\Delta \mathbf{E}V}\left(\beta\mathbf{E}V_{1}-c\right)dH\left(c\right)+\beta\int_{\Delta \mathbf{E}V}^{\infty}\mathbf{E}V_{0}dG\left(c\right)\\
 & =\Pi_{it}\left(\phi_{it},\mathbf{S}_{it}\right)+ \beta \mathbf{E}V_{1}\left[1-\exp(-\frac{\Delta\mathbb{E}V}{\gamma_{it}^{d}})\right]+\left(\Delta \mathbf{E}V+\gamma_{it}^{d}\right)\exp\left(-\frac{\Delta\mathbb{E}V}{\gamma_{it}^{d}}\right)\\
 & -\gamma_{it}^{d}+\mathbf{E}V{}_{0}\exp\left({-\frac{\Delta\mathbb{E}V}{\gamma_{it}^{d}}}\right)
\end{align*}
where $\Delta \mathbf{E}V=\beta\mathbf{E}V_{1}-\beta\mathbf{E}V_{0}$. Therefore the expression of $V_{d}$
can be simplified as 
\begin{equation}
V_{d}=\Pi_{it}\left(\phi_{it},\mathbf{S}_{it}\right)+\gamma_{it}^{d}\left(\exp\left(-\frac{\Delta \mathbf{E}V}{\gamma_{it}^{d}}\right)-1\right)+\mathbf{E}V_{1},\,\text{for }d\in\left\{ 0,\,1\right\} \label{Vd}
\end{equation}
In computing the value functions, $V_{d}$ is a $200$ by 1 vector
given capital, age, and industry. Let $p_{mn}=Pr\left(n_{t+1}=m,b_{t+1}=n|d_{t}=1\right)$,
for $m,\,n\in\left\{ 0,1\right\} $, and further denote $P_{mn}$
as the corresponding transition matrix of the productivity and $P_{0}$
as the transition matrix of productivity when $d_{t}=0$. then Equation
(\ref{Vd}) can be transformed as 
\begin{align}
V_{1} & =\Pi\left(\phi\right)-\beta\gamma^{1}\left(1-\exp\left(-\frac{\Delta EV}{\gamma^{1}}\right)\right)+\beta\left(\sum_{m}\sum_{n}p_{mn}P_{mn}\right)V_{1}\label{V1}\\
V_{0} & =\pi\left(\phi\right)-\beta\gamma^{0}\left(1-\exp\left(-\frac{\Delta EV}{\gamma^{0}}\right)\right)+\beta\left(\sum_{m}\sum_{n}p_{mn}P_{mn}\right)V_{1}\label{V0}
\end{align}
Denote $P_{1}=\beta\left(\sum_{m}\sum_{n}p_{mn}P_{mn}\right)$, then
\[
V_{1}=\left(I-\beta P_{1}\right)^{-1}\left[\pi\left(\phi\right)-\beta\gamma^{1}\left(1-\exp\left(-\frac{\Delta EV}{\gamma^{1}}\right)\right)\right]
\]
 because $\Delta EV=P_{1}V_{1}-P_{0}V_{0}$, it follows that:
\begin{align}
\Delta EV & =\left(I-\beta P_{0}\right)P_{1}\left(I-\beta P_{1}\right)^{-1}\left[\pi\left(\phi\right)-\beta\gamma^{1}\left(1-\exp\left(-\frac{\Delta EV}{\gamma^{1}}\right)\right)\right]\label{DEV}\\
 & \quad-P_{0}\left[\pi\left(\phi\right)-\beta\gamma^{0}\left(1-\exp\left(-\frac{\Delta EV}{\gamma^{0}}\right)\right)\right]\nonumber 
\end{align}
We use equation (\ref{DEV}) to solve for $\Delta EV$ and then we
use equations (\ref{V1}) and (\ref{V0}) to solve for $V_{1}$ and
$V_{0}$. Use $T_{\kappa}$ as the linear operator applied to $\Delta EV$,
it is easy to show that $T_{\kappa}$ is a contraction mapping. Then
$\Delta EV$ is a fixed point such that 
\[
T_{\kappa}\left(\Delta EV\right)=\Delta EV
\]
.Now we are in the position to use Newton-Kantorovich iterations.
First note that the Frech\'{e}t derivative of $T_{\kappa}$ with
respect to $\Delta EV$ is:
\begin{align}
T_{\kappa}' & =\frac{\partial T_{\kappa}\left(\Delta EV\right)}{\partial\Delta EV}\\
 & =\beta\left(\beta P_{0}-I\right)P_{1}\left(I-\beta P_{1}\right)^{-1}\exp\left[\text{diag}\left\{ -\frac{\Delta EV_{i}}{\gamma^{1}}\right\} \right]\nonumber \\
 & +\beta P_{0}\exp\left[\text{diag}\left\{ -\frac{\Delta EV_{i}}{\gamma^{0}}\right\} \right]\nonumber 
\end{align}
where 
\[
\text{diag}\left\{ -\frac{\Delta EV_{i}}{\gamma^{d}}\right\} =\left[\begin{array}{cccc}
-\frac{\Delta EV_{1}}{\gamma^{d}} & 0 & \cdots & 0\\
0 & -\frac{\Delta EV_{2}}{\gamma^{d}} & \cdots & 0\\
\vdots & \vdots & \ddots & \vdots\\
0 & 0 & \cdots & -\frac{\Delta EV_{n}}{\gamma^{d}}
\end{array}\right]
\]
Using the invertibility of $\left[I-T_{\kappa}'\right]$ the $n$th
iteration in the Newton-Kantorovich algorithm is 
\[
\Delta EV_{n+1}=\Delta EV_{n}-\left[I-T_{\kappa}'\right]^{-1}\left(I-T_{\kappa}\right)\left(\Delta EV_{n}\right)
\]
We set the tolerance as $e^{-6}$; the iteration stops when $\|\Delta EV_{n+1}-\Delta EV_{n}\|\leq e^{-6}$. 

\subsection{Outer loop: BHHH optimization algorithm}

The outer loop solves following problem:
\[
\max_{\kappa^{s},\kappa^{m}}\sum_{i}\sum_{t}l_{it}\left(\kappa,\,\Delta EV_{it}\right)
\]
 where 
\begin{align}
l_{it}\left(\kappa,\,\Delta EV_{it}\right) & =\log\left\{ d_{it}Pr\left(d_{it}=1|\kappa,d_{it-1}\right)+\left(1-d_{it}\right)Pr\left(d_{it}=0|\kappa,d_{it-1}\right)\right\} \label{llf}\\
 & =\log\left\{ \left(2d_{it}-1\right)\Pr\left(d_{it}=1|\kappa,d_{it-1}\right)+1-d_{it}\right\} \nonumber 
\end{align}
where $\Delta EV_{it}=\Delta EV\left(\phi_{it}\right)$ and 
\begin{align}
Pr\left(d_{it}=0|\kappa,d_{it-1}\right) & =1-Pr\left(d_{it}=1|\kappa,d_{it-1}\right)\\
 & =\exp\left\{ \frac{-\beta\Delta EV\left(\phi_{it}\right)}{\kappa^{m}d_{it-1}k_{it}+\kappa^{s}\left(1-d_{it-1}\right)k_{it}}\right\} \nonumber 
\end{align}
The basic parameter iteration under BHHH algorithm is:
\begin{align*}
\kappa_{n+1} & =\kappa_{n}+\lambda\underset{\equiv D\left(\kappa_{n}\right)}{\underbrace{\left[\sum_{i,t}\left(\frac{\partial l_{it}\left(\kappa_{n},\,\Delta EV_{it}\right)}{\partial\kappa_{n}}\right)\left(\frac{\partial l_{it}\left(\kappa_{n},\,\Delta EV_{it}\right)}{\partial\kappa_{n}'}\right)\right]^{-1}\left(\sum_{i,t}\frac{\partial l_{it}\left(\kappa_{n},\,\Delta\right)}{\partial\kappa_{n}}\right)}}
\end{align*}
From (\ref{llf}) we know that 
\begin{equation}
\frac{\partial l_{it}\left(\kappa_{n},\,\Delta EV_{it}\right)}{\partial\kappa_{n}'}=w_{it}\left[\frac{\partial Pr\left(d_{it}=1|\kappa,d_{it-1}\right)}{\partial\kappa_{n}^{s}},\,\frac{\partial Pr\left(d_{it}=1|\kappa,d_{it-1}\right)}{\partial\kappa_{n}^{m}}\right]
\end{equation}
where
\begin{align*}
w_{it} & =\frac{\left(2d_{it}-1\right)}{\left(2d_{it}-1\right)\Pr\left(d_{it}=1|\kappa,d_{it-1}\right)+1-d_{it}}\\
\frac{\partial Pr\left(d_{it}=1|\kappa,d_{it-1}\right)}{\partial\kappa_{n}^{s}} & =\beta\frac{\frac{\partial\Delta EV_{it}}{\partial\kappa_{n}^{s}}\gamma_{it}^{d_{it-1}}-\left(1-d_{it-1}\right)k_{it}\Delta EV_{it}}{\left(\gamma_{it}^{d_{it-1}}\right)^{2}\exp\left(\frac{\beta\Delta EV_{it}}{\gamma_{it}^{d_{it-1}}}\right)}\\
\frac{\partial Pr\left(d_{it}=1|\kappa,d_{it-1}\right)}{\partial\kappa_{n}^{m}} & =\beta\frac{\frac{\partial\Delta EV_{it}}{\partial\kappa_{n}^{m}}\gamma_{it}^{d_{it-1}}-d_{it-1}k_{it}\Delta EV_{it}}{\left(\gamma_{it}^{d_{it-1}}\right)^{2}\exp\left(\frac{\beta\Delta EV_{it}}{\gamma_{it}^{d_{it-1}}}\right)}
\end{align*}
where $\frac{\partial\Delta EV_{it}}{\partial\kappa^{s}}$ ($\frac{\partial\Delta EV_{it}}{\partial\kappa^{m}}$)
is the element in 1st (2nd) column such that the corresponding productivity
in the row is $\phi_{it}$. To finish the nested fixed point algorithm,
we need to compute the derivatives of the expected value function,
$\partial\Delta EV/\partial\gamma$. Applying the implicit theorem
to $T_{\kappa}\left(\Delta EV\right)=\Delta EV$, we get 
\[
\frac{\partial\Delta EV}{\partial\kappa}=\left[I-T_{\kappa}'\right]^{-1}\frac{\partial T_{\kappa}\left(\Delta EV\right)}{\partial\kappa}
\]
From (\ref{DEV}), we know that 
\begin{align*}
\frac{\partial T_{\kappa}\left(\Delta EV\right)}{\partial\kappa} & =\left[\begin{array}{cc}
\frac{\partial T_{\kappa}\left(\Delta EV\right)}{\partial\kappa^{s}} & ,\frac{\partial T_{\kappa}\left(\Delta EV\right)}{\partial\kappa^{m}}\end{array}\right]
\end{align*}
where 
\begin{align*}
\frac{\partial T_{\kappa}\left(\Delta EV\right)}{\partial\kappa^{m}} & =\beta k\left(\beta P_{0}-I\right)P_{1}\left(I-\beta P_{1}\right)^{-1}\left[1-\exp\left(\frac{-\Delta EV}{\gamma^{1}}\right)-\frac{\Delta EV}{\gamma^{1}}\varodot\exp\left(\frac{-\Delta EV}{\gamma^{1}}\right)\right]\\
\frac{\partial T_{\kappa}\left(\Delta EV\right)}{\partial\kappa^{s}} & =\beta kP_{0}\left[1-\exp\left(\frac{-\Delta EV}{\gamma^{0}}\right)-\frac{\Delta EV}{\gamma^{0}}\varodot\exp\left(\frac{-\Delta EV}{\gamma^{0}}\right)\right]
\end{align*}
are both $200$-by-$1$ vectors and $k$ is the exogenous state variable:
capital stock. We use $\varodot$ to denote the element-wise product.
To determine the step size $\lambda,$ we use secant iteration to
find the solution to $\partial f\left(\lambda\right)/\partial\kappa=0$,
where $f\left(\lambda\right)\equiv L\left(\kappa+\lambda D\left(\kappa\right)\right)$.
The iteration is given as:
\begin{equation}
\lambda_{m+1}=\lambda_{m}-\frac{\left(\lambda_{m}-\lambda_{m-1}\right)f'\left(\lambda_{m}\right)}{f'\left(\lambda_{m}\right)-f'\left(\lambda_{m-1}\right)}
\end{equation}
where 
\begin{equation}
f'\left(\lambda_{m}\right)=\sum_{i,t}\frac{\partial l_{it}\left(\kappa+\lambda_{m}D\left(\kappa_{n}\right),\,\Delta EV_{it}\right)}{\partial\kappa'}D\left(\kappa_{n}\right)
\end{equation}
This iteration determines the optimal step size $\lambda^{*}$. Finally,
the iteration stops when $\|\kappa_{n+1}-\kappa_{n}\|\leq e^{-6}$. 

\paragraph{Preliminary analysis on innovation and productivity}
In this subsection, I provide
preliminary evidence supporting that patent activities,especially on the extensive margin, is an important
channel for firms to realize the impact of R\&D on productivity and
hence profits. I use value added per employee as the measure of productivity,
denoted as $lp$. This variable is usually used
as a measure of labor productivity. In the econometric model, I use
the $lp_{t+1}$ as the dependent variable, and use $lp_{t}$, $rd_{t}$,
and its interactions with variables on patent activities as explanatory
variables. 

Table \ref{T8} reports the estimation results for different
explanatory variables. In column (1), I only use current productivity
and lagged binary indicator for R\&D as independent variables. As expected, the
coefficient of $rd_{t}$ is positive and significant. Column (2) shows the coefficient estimates
after we add the interaction term between R\&D status and invention
patents (denoted as $rd_{t}\times inpat_{t+1}$) to the empirical
model. We find that the coefficient estimate of $rd_{t}$ decreases
compared to that in column (1). The coefficient of $rd_{t}\times inpat_{t+1}$
is positive and statistically significant at 0.1\% significance level.
More importantly, the magnitude of the estimate of the coefficient
of $rd_{t}\times inpat_{t+1}$ is even larger than the estimate of
coefficient of $rd_{t}$, implying that patents is an important channel
through which the investment in R\&D contributes to the productivity
growth. In comparison, column (3) shows that the contribution of utility
patents to the realization of productivity-enhancing effect of R\&D
activities is not statistically significant. Columns (4) and (5) shows
that the interactive effect between invention patents and utility
patents may not be so important for past R\&D activities to increase
future productivity. To some extent, these results suggest that the
patent, especially invention patent, is an important channel through
which R\&D drives up the firm's future productivity. However,
we need to be cautious about interpreting our estimation results.
First, our measure of productivity is not clean as it contains information
on the capital intensity. Second, unobserved factors that affects
capital accumulation as well as firm's R\&D choice may cause endogeneity
problem to the preliminary specification. In all regressions, I control for industry, year, and industry-year fixed effects. 

As additional supporting evidence for that the extensive margin of
patents matters more in terms of realizing the productivity-promoting
effect of R\&D, I also try to re-estimate our models using a sub-sample
containing firms that filed at least one patent application. I
report the estimation results in Table \ref{T9}. As we can see from
the results, the estimates for coefficients of $rd_{t}\times inpat_{t+1}$
and $rd_{t}\times utpat_{t+1}$ are not statistically significant
though the coefficient estimate for $rd_{t}$ is significant in columns
(1) and (2). These results confirm that change in the extensive margin
of patent applications is more important in explaining the different
impacts of R\&D on productivity growth. 

\begin{center}
Table \ref{T8} here
\par\end{center}

\begin{center}
Table \ref{T9} here
\par\end{center}

The R\&D-patent and innovation-productivity linkages are two important components of our structural model. In light of the discussion above, we use 0-1 binary variables to measure the R\&D activities, invention patents applications, and utility patents applications. Specifically, we use $rd$, $n$, and $b$ to represent the investment in R\&D, application of invention
patents, and application of utility patents. $rd$ is equal to one
when the firm invests in R\&D; similar definitions are applied to
$n$ and $b$. Throughout our discussion, we do not consider design patents. 



\begin{table}[H]
\centering
\caption{Impact of R\&D and patents on productivity: extensive margin of patents}
\label{T8}
{
\def\sym#1{\ifmmode^{#1}\else\(^{#1}\)\fi}
\begin{tabular}{l*{5}{c}}
\hline\hline
           &\multicolumn{5}{c}{Dependent variable: $lp_{t+1}$} \\\cmidrule{2-6}
            &\multicolumn{1}{c}{(1)}&\multicolumn{1}{c}{(2)}&\multicolumn{1}{c}{(3)}&\multicolumn{1}{c}{(4)}&\multicolumn{1}{c}{(5)}\\

\hline
$lp_{t}$        &       0.598\sym{***}&       0.597\sym{***}&       0.597\sym{***}&       0.597\sym{***}&       0.597\sym{***}\\
            &   (0.00806)         &   (0.00806)         &   (0.00807)         &   (0.00806)         &   (0.00807)         \\
$rd_{t}$      &      0.0877\sym{***}&      0.0787\sym{***}&      0.0771\sym{***}&      0.0851\sym{***}&      0.0766\sym{***}\\
            &    (0.0124)         &    (0.0127)         &    (0.0129)         &    (0.0126)         &    (0.0129)         \\
$rd_{t}\times inpat_{t+1}$        &                     &       0.131\sym{***}&       0.123\sym{***}&                     &       0.133\sym{**} \\
            &                     &    (0.0346)         &    (0.0369)         &                     &    (0.0440)         \\
$rd_{t}\times utpat_{t+1}$        &                     &                     &      0.0313         &                     &      0.0424         \\
            &                     &                     &    (0.0365)         &                     &    (0.0435)         \\
$rd_{t}\times inpat_{t+1} \times utpat_{t+1} $     &                     &                     &                     &       0.124\sym{*}  &     -0.0430         \\
            &                     &                     &                     &    (0.0491)         &    (0.0774)         \\
\hline
\(N\)       &       11859         &       11859         &       11859         &       11859         &       11859         \\
\(R^{2}\)   &       0.531         &       0.532         &       0.532         &       0.531         &       0.532         \\
\hline\hline
\end{tabular}
}

\caption*{\small{}Note: all regression contain industry and industry-year fixed effects. Standard errors in parentheses.{*} \(p<0.05\), {**} \(p<0.01\), {***} \(p<0.001\)} {\small \par}
\end{table}


\begin{table}[H]
\centering
\caption{Impact of R\&D and patents on productivity: intensive margin of patents}
\label{T9}
{
\def\sym#1{\ifmmode^{#1}\else\(^{#1}\)\fi}
\begin{tabular}{l*{5}{c}}
\hline\hline
            &\multicolumn{5}{c}{Dependent variable: $lp_{t+1}$}\\\cmidrule{2-6}
            &\multicolumn{1}{c}{(1)}&\multicolumn{1}{c}{(2)}&\multicolumn{1}{c}{(3)}&\multicolumn{1}{c}{(4)}&\multicolumn{1}{c}{(5)}\\
\hline
 $lp_{t}$        &       .598\sym{***}&       .547\sym{***}&       .616\sym{***}&       .623\sym{***}&       .614\sym{***}\\
            &   (.00806)         &    (.0405)         &    (.0554)         &    (.0561)         &    (.0554)         \\
$rd_{t}$      &      .0877\sym{***}&       .188\sym{*}  &      0.0604         &       0.142         &      0.0147         \\
            &    (0.0124)         &    (0.0794)         &     (0.111)         &    (0.0969)         &     (0.135)         \\
$rd_{t}\times inpat_{t+1}$    &                     &     -0.0116         &      0.0589         &                     &      0.0613         \\
            &                     &    (0.0329)         &    (0.0417)         &                     &    (0.0432)         \\
$rd_{t}\times utpat_{t+1}$    &                     &                     &    -0.00437         &                     &      0.0119         \\
            &                     &                     &   (0.00375)         &                     &    (0.0183)         \\
$rd_{t}\times inpat_{t+1} \times utpat_{t+1} $  &                     &                     &                     &   .000016        &    -0.00128         \\
            &                     &                     &                     &  (.000126)         &   (.00147)         \\
\hline
\(N\)       &       11859         &         474         &         138         &         138         &         138         \\
\(R^{2}\)   &       0.531         &       0.540         &       0.686         &       0.680         &       0.687         \\
\hline\hline
\end{tabular}
}

\caption*{\small{}Note: all regression contain industry and industry-year fixed effects. Standard errors in parentheses.{*} \(p<0.05\), {**} \(p<0.01\), {***} \(p<0.001\)} {\small \par}
\end{table}